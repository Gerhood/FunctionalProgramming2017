%% -*- coding: utf-8 -*-
\documentclass{beamer}

%% -*- coding: utf-8 -*-
\usetheme{Boadilla} % default
\useoutertheme{infolines}
\setbeamertemplate{navigation symbols}{} 

\usepackage{etex}
\usepackage{alltt}
\usepackage{pifont}
\usepackage{color}
\usepackage[utf8]{inputenc}
\usepackage{german}
\usepackage{listings}
\usepackage{hyperref}
\hypersetup{colorlinks=true}
\usepackage[final]{pdfpages}
\usepackage{url}
\usepackage{arydshln} % dashed lines
\usepackage{tikz}
\usepackage{mathpartir}

\DeclareUnicodeCharacter{3BB}{\ensuremath{\lambda}}


\newcommand\cmark{\ding{51}}
\newcommand\xmark{\ding{55}}

\newcommand{\nat}{\mathbf{N}}

\usepackage[all]{xy}

%% new arrow tip for xy
\newdir{|>}{!/4.5pt/@{|}*:(1,-.2)@^{>}*:(1,+.2)@_{>}}

\newcommand\cid[1]{\textup{\textbf{#1}}} % class names
\newcommand\kw[1]{\textup{\textbf{#1}}}  % key words
\newcommand\tid[1]{\textup{\textsf{#1}}} % type names
\newcommand\vid[1]{\textup{\texttt{#1}}} % value names
\newcommand\Mid[1]{\textup{\texttt{#1}}} % method names

\newcommand\TODO[1][]{{\color{red}{\textbf{TODO: #1}}}}

\newcommand\String[1]{\texttt{\dq{}#1\dq{}}}

\newcommand\ClassHead[1]{%
  \ensuremath{\begin{array}{|l|}
      \hline
      \cid{#1}
      \\\hline
    \end{array}}}
\newcommand\AbstractClass[2]{%
  \ensuremath{\begin{array}{|l|}
      \hline
      \cid{\textit{#1}}
      \\\hline
      #2
      \hline
    \end{array}}}
\newcommand\Class[2]{%
  \ensuremath{\begin{array}{|l|}
      \hline
      \cid{#1}
      \\\hline
      #2
      \hline
    \end{array}}}
\newcommand\Attribute[3][black]{\textcolor{#1}{\Param{#2}{#3}}\\}
\newcommand\Methods{\hline}
\newcommand\MethodSig[3]{\Mid{#2} (#3): \,\tid{#1}\\}
\newcommand\CtorSig[2]{\Mid{#1} (#2)\\}
\newcommand\AbstractMethodSig[3]{\Mid{\textit{#2}} (#3): \,\tid{#1}\\}
\newcommand\Param[2]{\vid{#2}:~\tid{#1}}

\lstset{%
  frame=single,
  xleftmargin=2pt,
  stepnumber=1,
  numbers=left,
  numbersep=5pt,
  numberstyle=\ttfamily\tiny\color[gray]{0.3},
  belowcaptionskip=\bigskipamount,
  captionpos=b,
  escapeinside={*'}{'*},
  language=java,
  tabsize=2,
  emphstyle={\bf},
  commentstyle=\mdseries\it,
  stringstyle=\mdseries\rmfamily,
  showspaces=false,
  showtabs=false,
  keywordstyle=\bfseries,
  columns=fullflexible,
  basicstyle=\footnotesize\CodeFont,
  showstringspaces=false,
  morecomment=[l]\%,
  rangeprefix=////,
  includerangemarker=false,
}

\newcommand\CodeFont{\sffamily}

\definecolor{lightred}{rgb}{0.8,0,0}
\definecolor{darkgreen}{rgb}{0,0.5,0}
\definecolor{darkblue}{rgb}{0,0,0.5}

\newcommand\highlight[1]{\textcolor{blue}{\emph{#1}}}
\newcommand\GenClass[2]{\cid{#1}\texttt{<}\cid{#2}\texttt{>}}

\newcommand\Colored[3]{\alt<#1>{\textcolor{#2}{#3}}{#3}}

\newcommand\nt[1]{\ensuremath{\langle#1\rangle}}

\newcommand{\free}{\operatorname{free}}
\newcommand{\bound}{\operatorname{bound}}
\newcommand{\var}{\operatorname{var}}
\newcommand\VSPBLS{\vspace{-\baselineskip}}

\newcommand\IF{\textit{IF}}
\newcommand\TRUE{\textit{TRUE}}
\newcommand\FALSE{\textit{FALSE}}

\newcommand\IFZ{\textit{IF0}}
\newcommand\ZERO{\textit{ZERO}}
\newcommand\SUCC{\textit{SUCC}}
\newcommand\ADD{\textit{ADD}}
\newcommand\SUB{\textit{SUB}}
\newcommand\MULT{\textit{MULT}}
\newcommand\DIV{\textit{DIV}}

\newcommand\PAIR{\textit{PAIR}}
\newcommand\FST{\textit{FST}}
\newcommand\SND{\textit{SND}}

\newcommand\CASE{\textit{CASE}}
\newcommand\LEFT{\textit{LEFT}}
\newcommand\RIGHT{\textit{RIGHT}}

\newcommand\Encode[1]{\lceil#1\rceil}
\newcommand\Reduce{\stackrel\ast\rightarrow_\beta}

\newcommand\Nat{\textit{Nat}}
\newcommand\Bool{\textit{Bool}}
\newcommand\Pair{\textit{Pair}}
\newcommand\Tfun[1]{#1\to}

\newcommand\Tenv{A}
\newcommand\Lam[1]{\lambda#1.}
\newcommand\App[1]{#1\,}
\newcommand\Succ{\textit{SUCC}\,}
\newcommand\Let[2]{\textit{let}\,#1=#2\,\textit{in}\,}

\newcommand\calE{\mathcal{E}}
\newcommand\calU{\mathcal{U}}
\newcommand\calP{\mathcal{P}}
\newcommand\calW{\mathcal{W}}

\newcommand\GEN{\textit{gen}}
\newcommand\EFV[1]{\textit{fv} (#1)}
\newcommand\Dom[1]{\textit{dom} (#1)}

%%% Local Variables: 
%%% mode: latex
%%% TeX-master: nil
%%% End: 

%%% frontmatter
\input{frontmatter}
\subtitle{Test data generators}
\usepackage{tikz}


\begin{document}

\begin{frame}
  \titlepage
\end{frame}
%----------------------------------------------------------------------
\begin{frame}[fragile]
  \frametitle{An application of type classes an monads}
  \begin{block}<+->{Remember QuickCheck for testing}
    \begin{itemize}
    \item Automatic generation of test cases to test properties specified by the programmer
    \item So far restricted to properties on predefined types
    \end{itemize}
  \end{block}
  \begin{block}<+->{But really\dots}
    Test data can be generated for the instances of a type class \texttt{Arbitrary} (defined by QuickCheck) 
  \end{block}
  \begin{block}<+->{To extend the scope of QuickCheck\dots}
    \begin{itemize}
    \item we only need to write new instance of \texttt{Arbitrary}!
    \item (require the IO monad)
    \end{itemize}
  \end{block}
\end{frame}
%----------------------------------------------------------------------
\begin{frame}[fragile]
  \frametitle{An example}
\begin{verbatim}
prop_binomi :: Integer -> Integer -> Bool
prop_binomi a b = (a + b) ^ 2 == a ^ 2 + 2 * a * b + b ^ 2
\end{verbatim}
  can be checked
\begin{verbatim}
Main> quickCheck prop_binomi
+++ OK, passed 100 tests.
\end{verbatim}
\end{frame}
%----------------------------------------------------------------------
\begin{frame}[fragile]
  \frametitle{Arbitrary and Gen}
  \begin{block}<+->{Type class \texttt{Arbitrary}}
\begin{verbatim}
class Arbitrary a where
  arbitrary :: Gen a -- generate values of type a
  shrink :: a -> [a] -- shrink values of type a
\end{verbatim}
    Type \texttt{Gen a}: instructions for creating a random value of type \texttt{a} (a monad)
  \end{block}
  \begin{block}<+->{Functions for sampling a random generator}
\begin{verbatim}
sample   :: Show a => Gen a -> IO ()
sample'  :: Gen a -> IO [a]
generate :: Gen a -> IO a
\end{verbatim}
  \end{block}
\end{frame}
%----------------------------------------------------------------------
\begin{frame}[fragile]
  \frametitle{Sampling test data}
  Remember \texttt{sample'  :: Gen a -> IO [a]}
  \footnotesize
  \begin{itemize}[<+->]
  \item \texttt{Main> sample' arbitrary}
  \item \texttt{[(),(),(),(),(),(),(),(),(),(),()]}
  \item \texttt{Main> sample' (arbitrary :: Gen Bool)}
  \item \texttt{[True,False,False,False,False,True,True,False,False,True,False]}
  \item \texttt{Main> sample' (arbitrary :: Gen Int)}
  \item \texttt{[0,2,0,6,5,2,-12,9,-15,-2,20]}
  \item \texttt{Main> sample' (arbitrary :: Gen Int)}
  \item \texttt{[0,-1,0,5,3,-1,-11,-8,14,-10,-19]}
  \end{itemize}
\end{frame}
%----------------------------------------------------------------------
\begin{frame}[fragile]
  \frametitle{Building generators}
\begin{verbatim}
elements  :: [a] -> Gen a
oneof     :: [Gen a] -> Gen a
frequency :: [(Int,Gen a)] -> Gen a
listOf    :: Gen a -> Gen [a]
vectorOf  :: Int -> Gen a -> Gen [a]
choose    :: Random a => (a,a) -> Gen a
\end{verbatim}
  \begin{itemize}
  \item \texttt{Random} is a predefined class for generating random data
  \item (some experiments)
  \end{itemize}
\end{frame}
%----------------------------------------------------------------------
\begin{frame}[fragile]
  \frametitle{Generating a Suit}
\begin{verbatim}
data Suit = Spades | Hearts | Diamonds | Clubs
     deriving (Show, Eq)
\end{verbatim}
\end{frame}
%----------------------------------------------------------------------
\begin{frame}[fragile]
  \frametitle{Generating a Rank}
\begin{verbatim}
data Rank = Numeric Integer | Jack | Queen
          | King | Ace
          deriving (Show, Eq, Ord)
\end{verbatim}
\end{frame}
%----------------------------------------------------------------------
\begin{frame}[fragile]
  \frametitle{Generating a card}
\begin{verbatim}
data Card = Card { rank :: Rank, suit :: Suit }
     deriving (Show)
\end{verbatim}
\begin{itemize}
\item need to combine a generator for \texttt{Rank} and one for \texttt{Suit}
\item no provision in the QuickCheck library, but \dots
\end{itemize}
\end{frame}
%----------------------------------------------------------------------
\begin{frame}[fragile]
  \frametitle{\texttt{Gen} is a monad}
  \begin{itemize}
  \item \texttt{Gen a} is the type of instructions to generate random values of type \texttt{a}
  \item \texttt{IO a} is the type of instructions for IO operations with result \texttt{a}
  \item Both are monads $\Rightarrow$ use bind \texttt{>>=} to combine generators
  \item Alternatively, the \texttt{do} notation can be used with \texttt{Gen}
  \end{itemize}
\end{frame}
%----------------------------------------------------------------------
\begin{frame}[fragile]
  \frametitle{Examples}
  Generate 
  \begin{itemize}
  \item \texttt{Card}
  \item constant, twice
  \item even integers, non-negative integers
  \item \texttt{Hand}
  \end{itemize}
\end{frame}
%----------------------------------------------------------------------
\begin{frame}[fragile]
  \frametitle{Task: Check the generator}
  \begin{itemize}
  \item<+-> \texttt{Rank} contains useless values
  \item<.-> does its generator \texttt{rRank} only yield useful values?
  \end{itemize}
  \begin{block}<+->{Test it!}
\begin{verbatim}
validRank :: Rank -> Bool
validRank (Numeric n) = 2 <= n && n <= 10
validRank _ = True
\end{verbatim}
\begin{verbatim}
prop_all_validRank = forAll rRank validRank
\end{verbatim}
  \end{block}
\end{frame}
%----------------------------------------------------------------------
\begin{frame}[fragile]
  \frametitle{Checking properties of test data}
\begin{verbatim}
prop_all_valid_rank_collect r = collect r (validRank r)
\end{verbatim}
  \begin{itemize}
  \item \texttt{collect x} does not change the test
  \item collects values of \texttt{x} and creates a histogram
  \end{itemize}
\end{frame}
%----------------------------------------------------------------------
\begin{frame}[fragile]
  \frametitle{Task}
  \begin{itemize}
  \item Define a property that yields a histogram of generated \texttt{Hand}s
  \end{itemize}
\end{frame}
%----------------------------------------------------------------------
\begin{frame}
  \frametitle{Testing properties of \texttt{insert}}
  \onslide<+->{}
  \begin{block}<+->{Example}
  \begin{itemize}
  \item \texttt{insert x xs} inserts a value \texttt{x} in an ordered list \texttt{xs}
  \item the output should be ordered again (along with other properties)
  \item how do we test that?
  \end{itemize}
\end{block}
\end{frame}
%----------------------------------------------------------------------
\begin{frame}[fragile]
  \frametitle{First attempt}
\begin{verbatim}
prop_insert_1 :: Integer -> [Integer] -> Bool
prop_insert_1 x xs = isOrdered (insert x xs)
\end{verbatim}
\end{frame}
%----------------------------------------------------------------------
\begin{frame}[fragile]
  \frametitle{Second attempt}
\begin{verbatim}
prop_insert_2 :: Integer -> [Integer] -> Property
prop_insert_2 x xs = isOrdered xs ==> isOrdered (insert x xs)
\end{verbatim}
\end{frame}
%----------------------------------------------------------------------
\begin{frame}[fragile]
  \frametitle{Third attempt}
  \begin{block}<+->{A dedicated generator for sorted lists}
\begin{verbatim}
orderedList :: (Arbitrary a, Ord a) => Gen [a]
\end{verbatim}
    (How would you implement this generator?)
  \end{block}
  \begin{block}<+->{Usage}
\begin{verbatim}
prop_insert_3 x =
    forAll orderedList (\xs->isOrdered (insert x xs))
\end{verbatim}
  \end{block}
\end{frame}
%----------------------------------------------------------------------
\begin{frame}[fragile]
  \frametitle{Fourth attempt}
  \begin{block}<+->{A dedicated generator for sorted lists (defined by QuickCheck)}
\begin{verbatim}
data OrderedList a = Ordered [a]

instance (Ord a,Arbitrary a)
         => Arbitrary (OrderedList a) where
  arbitrary = orderedList
\end{verbatim}
  \end{block}
  \begin{block}<+->{Usage}
\begin{verbatim}
prop_insert_4 x (Ordered xs) = isOrdered (insert x xs)
\end{verbatim}
  \end{block}
\end{frame}
%----------------------------------------------------------------------
\begin{frame}
  \frametitle{Wrapup}
  \begin{alertblock}{Roll your own test data generators}
  \begin{itemize}
  \item populate the class \texttt{Arbitrary} with the types you want to generate
  \item generating managed by monad \texttt{Gen}
  \item conditional test generation
  \end{itemize}
  \end{alertblock}
\end{frame}

% \begin{frame}
%   \frametitle{Break Time --- Questions?}
%   \begin{center}
%     \tikz{\node[scale=15] at (0,0){?};}
%   \end{center}
% \end{frame}


\end{document}

%%% Local Variables: 
%%% mode: latex
%%% TeX-master: t
%%% End: 
